\section{Literature Review}

A matrix for example $\begin{pmatrix}
	1 & 0\\
	0 & 1
\end{pmatrix}$ can be also used in the document... Integer dui leo, fringilla eu ultrices tincidunt, sollicitudin nec sapien. Cras nec ligula non sem blandit sollicitudin sit amet vitae leo. Aenean accumsan ligula tincidunt, consectetur magna consectetur, porttitor enim. Nam quis volutpat ex. Cras sit amet finibus velit. Maecenas bibendum scelerisque lectus, at elementum urna faucibus ut. Suspendisse tempor tellus in lacus feugiat pretium. Nulla nec eros faucibus, facilisis mauris vitae, gravida nunc.

\subsection{Some subsection}
\lipsum[10]
\begin{lemma}
	Given two line segments whose lengths are $a$ and $b$ respectively there 
	is a real number $r$ such that $b=ra$.
\end{lemma}

\begin{proof}
	To prove it by contradiction try and assume that the statement is false,
	proceed from there and at some point you will arrive to a contradiction.
	\begin{align*}
	f(x) &= x^2\\
	g(x) &= \frac{1}{x}\\
	F(x) &= \int^a_b \frac{1}{3}x^3
	\end{align*}
\end{proof}
Cras sit amet finibus velit. Maecenas bibendum scelerisque lectus, at elementum urna faucibus ut. Suspendisse tempor tellus in lacus feugiat pretium. Nulla nec eros faucibus, facilisis mauris vitae, gravida nunc.

\subsection{Some subsection}
 Donec luctus volutpat orci. Aliquam purus erat, suscipit eget aliquam ut, maximus vel enim. Quisque semper imperdiet tellus in scelerisque. Nunc ornare non odio ut porta. Suspendisse dapibus vitae ante at condimentum. Nullam vel efficitur sapien. Nunc urna mi, luctus ut magna nec, viverra faucibus quam. In vehicula velit nec diam scelerisque pulvinar.
 
 \begin{theorem}
 	Given two line segments whose lengths are $a$ and $b$ respectively there 
 	is a real number $r$ such that $b=ra$.
 \end{theorem}
 
 \begin{proof}
 	To prove it by contradiction try and assume that the statement is false,
 	proceed from there and at some point you will arrive to a contradiction.
 	\begin{equation*}
 	 f(x) = \sum_{i=0}^{n} \frac{a_i}{1+x} 
 	\end{equation*}
 \end{proof}
